https://github.com/PhilJungschlaeger/rand_pixel_sampling/tree/master/source

\begin{frame}{Proximity-Direction-Interpolation}
\framesubtitle{Shadwoing}
The shadowing is implemented the way, I described it last time. The interpolation is able to determine which samples should, or should not have influence for a pixel.\\ %Voronoi is not a good solution, to decide about that. example?: xo|x
%Example picture of new tech?
\end{frame}

\begin{frame}{Proximity-Direction-Interpolation}
\framesubtitle{Angle}
The Angle-influence is implemented in a different way. Last time i thought, the shadow area-pie-parts, should decide about the influence of a sample. But this is kind of unfair. Because the proximity gets a influence , with the power of 2 (\(\pi \cdot r^2\)), but the angle hs only a linear influence. \\
So:  \((r-d) \cdot (\alpha)\)\\
r: the sum of all distances\\
d: the distance of the sample towards the pixel\\
\(\alpha\): the angle of influence\\
\end{frame}

\begin{frame}{Proximity-Direction-Interpolation}
\framesubtitle{Angle}
%example image
\end{frame}

\begin{frame}{Proximity-Direction-Interpolation}
\framesubtitle{Proximity: new-radius}
I realised, that my old calcutlation of the  proiximity influence was a bit unfair for close points and too friendly for far points:\\
influence: \((r-d)\\
with r: the sum of all distances\\
I changed the definition of r:\\
r= the closest+the farest distance\\
This is the smallest imaginable r-value:\\
If you have two influencing points, with nearly the same proximity towards the pixel. Both only get the same influence with this formular.\\
\end{frame}

\begin{frame}{Proximity-Direction-Interpolation}
\framesubtitle{Proximity: new-radius}
%old radius pic
\end{frame}

\begin{frame}{Proximity-Direction-Interpolation}
\framesubtitle{Proximity: new-radius}
%new radius pic
\end{frame}

\begin{frame}{todo}
Next week i will come to germany and hopefully to the project meeting.\\
I would like to interpolate in time: \\
-Nabile's idea in time.\\
-Voronoi-3D\\
I want to do a big quality assesment of my proximity-direction-interpolation and a comparison to voronoi and splatting and combinations of all of them after new year.\\
\end{frame}



\begin{frame}{Proximity-Interpolation}
\framesubtitle{Development}
Last week code was horribly slow.. \\
\(\rightarrow\)Interpolating a simple Full-HD Picture:\\
\begin{itemize}
\item a few samples(200samples): fast(a minute)
\item increasing number of samples(10percent): slow(3hours)
\end{itemize}

I made a \textbf{runtime-algorithm-analysis}. According to my analysis, the algorithm should get
faster to the power of 2 when you use more samples.. So I probably miss-interpreted
some tasks:\\
\begin{itemize}
\item swithc from vector to list(easy insertion)
\item implement a bucketing system!
\end{itemize}
\end{frame}

\begin{frame}{Proximity-Interpolation}
\framesubtitle{Development}
Last week code was horribly slow.. \\
\(\rightarrow\)Interpolating a simple Full-HD Picture:\\
\begin{itemize}
\item a few samples(200samples): fast(a minute)
\item increasing number of samples(10percent): slow(3hours)
\end{itemize}

I made a \textbf{runtime-algorithm-analysis}. According to my analysis, the algorithm should get
faster to the power of 2 when you use more samples.. So I probably miss-interpreted
some tasks:\\
\begin{itemize}
\item swithc from vector to list(easy insertion)
\item implement a bucketing system!
\end{itemize}
\end{frame}

\begin{frame}{Proximity-Interpolation}
\framesubtitle{Bucketing System}
Pre-Sorting:\\
To sort a Pixel(x,y) in bucket(N), i developed this formular:\\
\(N=x*X/SIZEX+X*(y*Y/SIZEY);\)\\
With:\\
\begin{itemize}
\item X: the number of buckets in x-dimension
\item Y: the number of buckets in y-dimension
\item SIZEX: number of pixel in x-dimension
\item SIZEY: number of pixel in y-dimension
\end{itemize}
The pre-sorting does not take any extra time, because you have to iterate(read) about the random
samples anyway.\\ \(\rightarrow\) it does not make sense, to include the bucketing to the fileformat\\
\end{frame}

\begin{frame}{Proximity-Interpolation}
\framesubtitle{Bucketing System}
Find closest Points with buckets:\\
To find the closest samples towards a not sampled pixel, we don't want to iterate through all samples.\\
\begin{enumerate}
\item check all the samples, that are in the same bucket, as the our Pixel, that is to be interpolated
\item if we did not find enough closest samples or the farest sample is more far away than the next bucket-border \(go for the closest bucket\)
\end{enumerate}
\end{frame}
We need to remember which buckets are already visited.\\
And we need to determine, whether a bucket is part of the image..\\
My bucketing system gives an immense speed-up! But how many buckets make sense according to the number of influencing points and the number of samples?\\
Unfortunately I just noticed a buck at the borders of the buckets.\\
\(\rightarrow\) I will repair that\\
\end{frame}


\begin{frame}{Proximity-Interpolation}
\framesubtitle{Development}
After repairing the bucket-system, I want to compute an array of pictures with:
\begin{itemize}
\item the number of samples
\item the number of influencing samples
\item the power of samples influence
\end{itemize}
Some example pictures:\\
\end{frame}

\begin{frame}{Proximity}
\begin{center}
%\includegraphics[scale=0.4]{sq_1_1.png}
\end{center}
With 10 percent of data, only cloest point.\\
You see it is the same as voronoi!
\end{frame}

\begin{frame}{Proximity}
\begin{center}
%\includegraphics[scale=0.4]{sq_10_30.png}
\end{center}
With 10 percent of data, 10 cloesest points, power of 30.\\
Maybe too much influencing points?\\
\end{frame}

\begin{frame}{Proximity}
\begin{center}
%\includegraphics[scale=0.4]{sq_3_3.png}
\end{center}
With 10 percent of data, 3 cloesest points, power of 3.\\
\\
\end{frame}

\begin{frame}{Proximity}
\begin{center}
%\includegraphics[scale=0.4]{sq_3_30.png}
\end{center}
With 10 percent of data, 3 cloesest points, power of 30.\\
Starts to look like voronoi, because of the great power!\\
\end{frame}

//using the proximity idea in a thrid dimension: time
//video-test

\begin{frame}{Proximity-Interpolation}
\framesubtitle{plus a third dimension: time}
In a way, we could use the idea of my private photography project, for our project.\\
Instead of only using x and y dimension to interpolate, we could use 3 dimensions(x,y,t).\\
In a movie, we could use the information of the previous picture in the next picture...\\
I would like to take a look at that and sample our first random movie!\\
\end{frame}

\begin{frame}{time}
This week, i will tramp from Siurana(Barcelona) to Granda.\\
\(\rightarrow\) Deepending on how good, that works, I will  be working
more or less.
\end{frame}
